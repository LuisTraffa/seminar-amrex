\documentclass[12pt, a4paper]{article}

\usepackage[a4paper, left=2.5cm, right=2.5cm, top=2.5cm, bottom=2.5cm]{geometry}
\usepackage[utf8]{inputenc}
\usepackage[T1]{fontenc}
\usepackage{graphicx}
\usepackage{hyperref}
\usepackage{amsmath}
\usepackage{amsfonts}
\usepackage{amssymb}
\usepackage{caption}
\usepackage{subcaption}
\usepackage{booktabs}
\usepackage{siunitx}
\usepackage{url}
\usepackage[backend=biber, style=ieee, autocite=inline]{biblatex}
\addbibresource{references/references.bib}

\title{Efficient Programming of HPC Systems: A Study on the AMReX Framework}
\author{Luis Traffa}
\date{\today}

\begin{document}

\maketitle
\tableofcontents

\clearpage

\section{Introduction}
\subsection{Area and Problem}

HPC systems are used to quikcly solve mathematical problems. Often times 
these problems are near impossible to solve in a reasonable amount of time.
Therefore, these systems employ techniques from numerival analysis to give approximate solutions,
instead of exact ones. 

One such technique is called adaptive mesh refinement (AMR). It is used during simulations of various natural 
phenomena, such as weather, climate, and astrophysics. The idea is to use a coarse mesh to simulate the
phenomena, and then refine the mesh in areas of interest. This allows for a more accurate simulation, while
keeping the computational cost low.


\subsection{AMReX Framework}

The AMReX framework is a software library that provides a set of tools to implement AMR algorithms.
It is developed by the Center for Computational Sciences and Engineering (CCSE) at the Lawrence Berkeley National Laboratory (LBNL).
The framework is written in C++ and is open source. It is used in a variety of projects, such as the
Exascale Computing Project (ECP) and the Energy Exascale Earth System Model (E3SM).

This framework aids in developing block-strucutred AMR algorithms by providing a set of tools to
solve systems of partial differential equations and manage data structures and the parallelization of the algorithms.
It is designed to be highy performant, easy-to-use, flexible and portable across different architectures.
Hence, most of the parallelization was abstarcted away, to enable the users to focus on the mathematical and physical
aspects of their problems. However, users can still access lower level features if the need arises.




\section{Background}
\subsection{Vocabulary}
\begin{itemize}
\item Meshes: A mesh is a geometrical representation of a physical domain, often a space, 
which is broken down into small, discrete elements. These elements can be in various shapes 
such as triangles, rectangles, or hexagons in 2D, and tetrahedra, prisms, or hexahedra in 3D. 
These small units are interconnected by nodes or vertices, forming a network that covers the 
entire domain. Mesh is typically used in numerical simulations, such as Finite
Element Method (FEM), Finite Volume Method (FVM), or Finite Difference Method (FDM), to 
discretize a continuum domain for solving complex mathematical problems.
\item Partial Differential Equations (PDEs): A PDE is a type of differential equation that 
contains unknown multivariable functions and their partial derivatives. PDEs are used to 
formulate problems involving functions of several variables, and are prevalent in physics 
and engineering. They are often used to describe wave propagation, heat diffusion, fluid flow, 
or quantum mechanics, amongst others.
\item Mesh Simulation: Mesh simulation involves the use of computational methods to 
solve equations over the domain represented by the mesh. This can involve a variety of 
different equations depending on the problem at hand, but often involves the solution 
of PDEs. The domain of interest is discretized into a mesh, and the equations are 
solved at each node or cell of the mesh. The goal of the simulation is often to predict 
physical behavior or solve complex engineering problems.
\item (Adaptive) Mesh Refinement: Mesh refinement, or adaptive mesh refinement (AMR), is 
a computational technique used to adaptively refine a computational mesh. The concept 
behind mesh refinement is to use fine mesh (small cells) in areas where the solution has 
high gradients or requires high accuracy, and coarse mesh (large cells) in areas where 
the solution is smooth or less important. This technique allows the efficient use of 
computational resources, improving the accuracy of the solution while reducing 
computational cost.
\end{itemize}

\subsection{Meshes and PEDs}

The relationshiop between meshes and PEDs is at the center of many numerical methods 
in computational physics and engineering. Therefore it is important to understand how 
the concepts relate to one another. 

\subsection{Programming Models}
\subsection{Algorithms}



\subsection{Hardware Architectures}

\section{AMReX: Approach}
\subsection{Introduction to AMReX}



\subsection{Efficiency Techniques}

Several techniques for efficient usage of resources are used in AMReX.

\begin{itemize}
    \item Adaptive Mesh Refinement: AMReX uses a block-structured AMR, which selectively 
    refines regions of the computational grid that require more resolution.
    This technique reduces computational cost by allocating resources only where needed.
    \item Parallelization: AMReX has strong support for parallelization with both MPI (Message Passing Interface) 
    and OpenMP, allowing it to run efficiently on distributed-memory and shared-memory systems. 
    It uses a combination of space-filling curve (SFC) based algorithms and two-level hierarchical 
    algorithms for parallelization.
    \item Load Balancing: AMReX uses a dynamic load balancing algorithm to ensure that computational 
    work is evenly distributed across all available processors, thus minimizing idle time and enhancing 
    overall performance.
    \item Vectorization: AMReX supports the use of modern many-core and multi-core architectures 
    and uses vectorization to exploit these architectures to the fullest.
    \item Asynchronous I/O: AMReX supports parallel I/O, which is particularly useful when dealing with large datasets.
\end{itemize}



\subsection{Portability Across Architectures}

One of AMReX's strengths is its ability to provide portability across a wide variety 
of computing architectures. It has been successfully deployed on a variety of high-performance 
computing systems, including CPUs and GPUs, and is designed to work seamlessly 
with upcoming exascale architectures.

\section{Evaluation}
\subsection{Publication 1}
\subsection{Publication 2}
\subsection{Publication 3}
\subsection{Publication 4}

\section{Discussion}
\subsection{Advantages of AMReX}
\subsection{Drawbacks of AMReX}
\subsection{Comparison with Other Approaches}

\section{Summary and Outlook}
\subsection{Key Findings}
\subsection{Future Research Directions}

\end{document}
